\documentclass[eq,capa]{texufpel}

% -------------------------------
% PACOTES BÁSICOS
% -------------------------------

\usepackage[utf8]{inputenc} % Permite acentuação direta (á, é, ç, etc.)
\usepackage[T1]{fontenc}   % Melhora a saída de fontes em PDF
\usepackage{graphicx}      % Inserção de figuras

% Configurações de hyperlinks
\hypersetup{
    hidelinks,   % Remove caixas e cores dos links
    unicode=true,% Permite acentuação nos bookmarks
    linktoc=all  % Links no título e número do sumário
}

% -------------------------------
% DADOS INSTITUCIONAIS
% -------------------------------

\unidade{Centro de Desenvolvimento Tecnológico}
\programa{Programa de Pós-Graduação em Computação}
\curso{Ciência da Computação}

% Versões em inglês (obrigatórias em pós-graduação)
\unidadeeng{Technology Development Center}
\programaeng{Postgraduate Program in Computing}
\cursoeng{Computer Science}

% -------------------------------
% DADOS DO TRABALHO
% -------------------------------

\title{Título da sua proposta de Qualificação}

% Autor
\author{Aguiar}{Marilton Sanchotene de}

% Orientador, coorientador e colaborador
\advisor[Prof.~Dr.]{Aguiar}{Marilton Sanchotene de}
\coadvisor[Prof.~Dr.]{Aguiar}{Marilton Sanchotene de}
\collaborator[Prof.~Dr.]{Aguiar}{Marilton Sanchotene de}

% -------------------------------
% PALAVRAS-CHAVE
% -------------------------------

% Português: letras minúsculas, separadas por ;
% a última termina com .
\keyword{aprendizado de máquina;}
\keyword{inteligência artificial;}
\keyword{processamento de dados;}
\keyword{classificação supervisionada.}

% Inglês
\keywordeng{machine learning;}
\keywordeng{artificial intelligence;}
\keywordeng{data processing;}
\keywordeng{supervised classification.}

\begin{document}

% Caso a orientadora seja mulher, descomente:
% \renewcommand{\advisorname}{Orientadora}
% \renewcommand{\coadvisorname}{Coorientadora}

\maketitle
\sloppy

% -------------------------------
% ELEMENTOS PRÉ-TEXTUAIS (OPCIONAIS)
% -------------------------------

\begin{dedicatoria}
Dedique este trabalho a pessoas ou instituições que foram importantes
na sua trajetória acadêmica ou pessoal.
\end{dedicatoria}

\begin{agradecimentos}
Agradeça a orientadores, familiares, colegas, agências de fomento
e demais pessoas que contribuíram direta ou indiretamente
para o desenvolvimento do trabalho.
\end{agradecimentos}

\begin{epigrafe}
Insira aqui uma citação curta que represente o espírito do trabalho.\\
{\sc --- Autor da citação}
\end{epigrafe}

% -------------------------------
% RESUMOS
% -------------------------------

\begin{abstract}
O resumo deve apresentar de forma clara e objetiva o problema de pesquisa,
os objetivos, a metodologia proposta e os principais resultados esperados.
Deve conter no máximo 500 palavras e ser escrito em parágrafo único,
sem citações ou referências bibliográficas.
\end{abstract}

\begin{englishabstract}{Title of the Work in English}
The abstract in English must be a faithful translation of the Portuguese
abstract, following the same structure and content, with a maximum
of 500 words.
\end{englishabstract}

% -------------------------------
% LISTAS E SUMÁRIO
% -------------------------------

\listoffigures
\listoftables

\begin{listofabbrv}{ABNT}
  \item[ABNT] Associação Brasileira de Normas Técnicas
  \item[IA] Inteligência Artificial
  \item[ML] Machine Learning
\end{listofabbrv}

\tableofcontents

% -------------------------------
% TEXTO PRINCIPAL
% -------------------------------

\chapter{Introdução}

Apresente o contexto geral da pesquisa, a motivação do estudo,
a relevância acadêmica e prática do tema e uma visão geral
do problema abordado.

\section{Contextualização}

Descreva o cenário em que o problema está inserido,
incluindo conceitos fundamentais e trabalhos iniciais
relacionados ao tema de pesquisa~\cite{Moore:1979:MAI,Aguiar:2005}.

\section{Problema de Pesquisa}

Delimite claramente o problema que será investigado,
indicando lacunas existentes na literatura e justificando
a necessidade do estudo~\cite{vonNeumann:1966:TSR}.

\chapter{Desenvolvimento}

Este capítulo deve detalhar a fundamentação teórica,
a metodologia proposta, os materiais e métodos,
bem como a organização das etapas do trabalho.

\section{Metodologia}

Explique o tipo de pesquisa, os métodos utilizados,
as ferramentas, os dados e os procedimentos adotados.

\begin{table}[ht]
\caption{Exemplo de Tabela}
\label{tabela2}
\centering
\begin{tabular}{p{4cm}p{5cm}p{6cm}}
\hline
Item & Descrição & Observação \\
\hline
A & Exemplo & Informação adicional \\
B & Exemplo & Informação adicional \\
\hline
\end{tabular}
\begin{flushleft}
{\small Fonte: Elaborada pelo autor.}
\end{flushleft}
\end{table}

\begin{figure}[htbp]
\caption{Exemplo de Figura}
\centering
\includegraphics[scale=.4]{figura}
\begin{flushleft}
{\small Fonte: Elaborada pelo autor.}
\end{flushleft}
\label{figura}
\end{figure}

\chapter{Conclusão}

Apresente as considerações finais do trabalho,
retomando os objetivos propostos, destacando as contribuições,
limitações do estudo e sugestões para trabalhos futuros.

% -------------------------------
% REFERÊNCIAS
% -------------------------------

\bibliographystyle{abnt}
\bibliography{bibliografia}

% -------------------------------
% APÊNDICES E ANEXOS
% -------------------------------

\apendices
\chapter{Exemplo de Apêndice}
Material elaborado pelo próprio autor.

\anexos
\chapter{Exemplo de Anexo}
Material produzido por terceiros (leis, documentos, manuais, etc.).

\end{document}
